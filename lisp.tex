\documentclass[12pt,letterpaper]{article}

\usepackage{amsmath}
\usepackage{paralist}

\begin{document}

\begin{center}
  LISP Programming Worksheet

  Created by Christopher Cooper and Sam Craig for the West Lafayette High School ACSL Club of 2014--2015
\end{center}

\textbf{You may not need to know all of this material for the next contest!}
Problems 1 and 2 are representative of problems you may see on contest \#2.
Problems 3 through 5 are representative of problems you may see on contest \#4.

\bigskip
\noindent \textbf{Questions}

Evaluate the following LISP code.

\begin{enumerate}

\item
\begin{verbatim}
(DIV 24 (SUB 2 5))
\end{verbatim}

\item
\begin{verbatim}
(ADD (MULT 48 40)
     (SUB (SQUARE 9) 1)
     (MULT 3 5))
\end{verbatim}

\item What is the final returned value of the program? 
\begin{verbatim}
(SETQ a 'hello)
(ATOM a)
\end{verbatim}

\item What is the final value of \texttt{b} after this program is run?
\begin{verbatim}
(SETQ b '((what) is a list "?"))
(CONS (CAR b) (CDR b))
(SETQ b (CONS (CAR (CAR b)) (CDR b)))
\end{verbatim}

\item What is the final returned value of the program?
\begin{verbatim}
(SET 'c '(21 22 23 50))
(SETQ d (CONS 33 (CDR (CDR c)))
(REVERSE d)
\end{verbatim}

\end{enumerate}

\pagebreak

\noindent \textbf{Answers}

\begin{enumerate}

\item
\begin{verbatim}
(DIV 24 -3)
-8
\end{verbatim}

\item ~ %% \vspace{-20pt}
  \\ \begin{tabular}{p{5in}}
\begin{verbatim}
(ADD (MULT 48 40) (SUB 81 1) (MULT 3 5))
(ADD 1920 80 15)
2015
\end{verbatim}
\end{tabular}

\item ~ %% \vspace{-10pt}
  \\ \begin{tabular}{l l}
\texttt{(SETQ a 'hello)} & Set variable \texttt{a} to symbol \texttt{hello}. \\
\texttt{(ATOM a)}        & Returns true, since \texttt{a} is not a list.
  \end{tabular}

\item ~ %% \vspace{-10pt}
  \\ \begin{tabular}{l p{3.2cm}}
    \texttt{(SETQ b '((what) is a list "?"))} & Set variable \texttt{b} to list containing elements:
    \begin{inparaenum}
    \item A list containing one element, the symbol \texttt{what}.
    \item The symbol \texttt{is}.
    \item The symbol \texttt{list}.
    \item The string ``?''.
    \end{inparaenum}
    \\
    \texttt{(CONS (CAR b) (CDR b))}        & Creates the list \texttt{((what) "?")}. \\
    \texttt{(SETQ b (CONS (CAR (CAR b)) (CDR b)))} & Changes the value of the first element in \texttt{b} to the first element in the list \texttt{(what)}, the symbol \texttt{what}.
  \end{tabular}

  Therefore, the value of \texttt{b} is \texttt{(what is a list "?")}.

\item ~ %% \vspace{-10pt}
  \\ \begin{tabular}{l p{4.2cm}}
    \texttt{(SET 'c '(21 22 23 50))} & Set variable \texttt{c} to list containing elements 21, 22, 23, and 50. \\
    \texttt{(SETQ d (CONS 33 (CDR (CDR c)))} & Set variable \texttt{d} to list containing elements 33 and 50. \\
    \texttt{(REVERSE d)} & Returns \texttt{(50 33)}.
  \end{tabular}
  
\end{enumerate}

\end{document}

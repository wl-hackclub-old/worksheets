\documentclass[12pt,letterpaper]{article}

\begin{document}

\begin{center}
  Computer Number Systems Worksheet

  Created by Sam Craig and the West Lafayette ACSL Club of 2014--2015
\end{center}

A subscript after a number designates what base it is in.
For example, $20_{10}$ is 20 in base ten, but usually base ten is assumed and left out.
$10_{16}$ is 10 in base 16, or 16 in base ten.
A \textbf{bit} is a \textbf{bi}nary digi\textbf{t}.

\bigskip
\noindent \textbf{Questions}

\begin{enumerate}

\item Convert $5250_{10}$ to hexadecimal.

\item Convert $2036_{10}$ to binary.

\item Convert $15\textrm{F}_{16}$ to octal.

\item Convert $101000100011_2$ to hexadecimal.

\item Evaluate $21\textrm{A}7_{16} - 110_{16}$ in hexadecimal.

\item Evaluate $10531_8 + 12414_8$ in octal.

\item Evaluate $1101111_2 - 111011_2$ in binary.

\item Evaluate $3\textrm{C}0_{16} + 340_8$ in hexadecimal.

  %% This should be replaced with a non-packet problem in the future.
\item In the ACSL computer, each ``word'' of memory
  contains 20 bits representing 3 pieces of information.
  The most significant 6 bits represent Field A; the next
  11 bits, Field B; and the last 3 bits represent Field C.
  For example, the 20 bits comprising the ``word''
  $18149_{16}$ has fields with values of $6_{16}$, $29_{16}$,
  and $1_{16}$. What is Field B in $\textrm{E}1\textrm{B}7\textrm{D}_{16}$? (Express your answer as
  a base 16 number.)

\end{enumerate}

\pagebreak

\noindent \textbf{Answers}

%% How these answers are retrieved should be explained.
\begin{enumerate}

\item $1482_{16}$

\item $11111110100_2$

\item $537_8$

\item $\textrm{A}23_{16}$

\item $2097_{16}$

\item $23145_{8}$

\item $110100_2$

\item $2240_8$

\item $36\textrm{F}_{16}$

\end{enumerate}

\end{document}
